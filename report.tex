\RequirePackage{plautopatch}
\documentclass[uplatex,dvipdfmx,11pt]{jsarticle}

% 余白設定
\usepackage[top=15truemm,bottom=15truemm,left=30truemm,right=30truemm]{geometry}

% MSフォントを指定
% https://qiita.com/zr_tex8r/items/a13c195d42b7fca69378
\usepackage[ms]{pxchfon}

% 1行当たりの文字数、ページ当たりの行数の指定
% https://rion778.hatenablog.com/entry/20091002/1254482262
\makeatletter
\def\mojiparline#1{
    \newcounter{mpl}
    \setcounter{mpl}{#1}
    \@tempdima=\linewidth
    \advance\@tempdima by-\value{mpl}zw
    \addtocounter{mpl}{-1}
    \divide\@tempdima by \value{mpl}
    \advance\kanjiskip by\@tempdima
    \advance\parindent by\@tempdima
}
\makeatother
\def\linesparpage#1{
    \baselineskip=\textheight
    \divide\baselineskip by #1
}

% 参考文献
\usepackage[maxnames=5,sortlocale=ja_JP]{biblatex}
\addbibresource{index.bib}

%% % 日本語文献で最終著者の前にandが入らないようにするため
% https://gist.github.com/idiotWu/4879a88a3e34618cc8215a43ba2e9fbd
% 日本語文献サポート
\AtEveryBibitem{
  % langid={Japanese} で識別
  \iffieldequalstr{langid}{Japanese} {
    % コンマで区切り
    \DeclareDelimFormat{finalnamedelim}{%
      \ifnumgreater{\value{liststop}}{2}{\finalandcomma}{}%
      \addspace\multinamedelim
    }
    %% % 名前を「姓 名」の順に
    %% \DeclareNameFormat{default}{%
    %%   \usebibmacro{name:delim}{\namepartfamily\namepartgiven}%
    %%   \usebibmacro{name:hook}{\namepartfamily\namepartgiven}%
    %%   \namepartfamily\bibnamedelimc\namepartgiven%
    %%   \usebibmacro{name:andothers}
    %% }
    % 年月日に参照
    %% \DeclareFieldFormat{urldate}{%
    %%   (\thefield{urlyear}-\thefield{urlmonth}-\thefield{urlday}~に参照\isdot)
    %% }
    % カギカッコでタイトルを囲む
    %% \DeclareFieldFormat*{title}{「#1」}
    %% \DeclareFieldFormat[book]{title}{『#1』}
  }{}
}


% カウンターの出力を全角数字にする方法
% https://oku.edu.mie-u.ac.jp/tex/mod/forum/discuss.php?d=58
\makeatletter
\def\@zenkakusuuji#1{%
\ifcase#1 0\or 1\or 2\or 3\or 4\or 5\or 6\or 7\or 8\or 9\else\relax\fi%
}%
\def\zenkakusuujihenkan#1{%
\edef\temp{#1}%
\let\@zenkaku=\empty%
\expandafter\@tfor\expandafter\@singlenumber\expandafter:\expandafter=\temp\do{%
\let\@tmpzenkaku=\@zenkaku%
\edef\@zenkaku{\@tmpzenkaku\@zenkakusuuji\@singlenumber}%
}%
\@zenkaku%
}%
\newcounter{bangou}%
\setcounter{bangou}{1}%
\makeatother


% sectionの文字サイズ変更
\usepackage{titlesec}
\titleformat{\section}{}{\zenkakusuujihenkan{\thesection}.}{0zw}{}
\titlespacing{\section}{0zw}{1zw}{0zw}
\titleformat{\subsection}{}{\zenkakusuujihenkan{\thesection}.\zenkakusuujihenkan{\arabic{subsection}}}{1zw}{}
\titlespacing{\subsection}{0zw}{0zw}{0zw}
\titleformat{\subsubsection}{}{\zenkakusuujihenkan{\thesection}.\zenkakusuujihenkan{\arabic{subsection}}.\zenkakusuujihenkan{\arabic{subsubsection}}}{1zw}{}
\titlespacing{\subsubsection}{0zw}{0zw}{0zw}

% ページ番号を消す
\pagestyle{empty}

%%%%%%%%%%%%%%%%%%%%%%%%%%%%%%%%%%%%%%%%%%%%%%%%%%%%%%%%%%%%%%%%%%%%%%

\begin{document}
% 一行あたり文字数の指定
\mojiparline{43}
% 1ページあたり行数の指定
\linesparpage{63}

% 段落最初の字下げ
\parindent = 1zw

\noindent
様 式 C-19、F-19-1、Z-19(共通)\\ \\

\section{研究開始当初の背景}

\section{研究の目的}

\section{研究の方法}

\section{研究成果}

------------------------------------------------------------

↑Word版のテンプレートと同じ記載内容

------------------------------------------------------------

\section{簡単な説明}

\subsection{フォント}

Word版と同じ見た目にするために、フォントはMS明朝を使用します。
msmincho.ttc というファイルをLaTeXから見える位置(例えば、ソースのディレクトリ)に置いてください。
MSゴシックのフォントファイル(msgothic.ttc)が無いとコンパイル時に怒られるようなのですが、使っていなかったら実害は無いみたいです。

\subsection{参考文献}

参考文献の例です~\cite{dummy_ref1, dummy_ref2}。

なんとなく、参考文献にはBibLaTeXを使っています。
そのため、コンパイル時は
\begin{verbatim}
# pdfuplatex report.tex
# biber report
# pdfuplatex report.tex
# pdfuplatex report.tex
\end{verbatim}
の要領でコンパイルしてください。
その影響(?)で、bibファイル中の日本語の参考文献には
\begin{verbatim}
  langid = {Japanese},
\end{verbatim}
を入れないと、最終著者の前にandが入ります。

まあ、BibTeXに変更してもらっても良いと思います。


\subsection{縦横の文字数を数えるためのテストパターン}
多分使わなくても良いと思いますが、Word版と文字数、行数を比較するときには、dummy.texを読み込んでください。
\begin{verbatim}
\clearpage
\noindent
123456789012345678901234567890123456789012345678901234567890123456789012345678901234567890

\clearpage
\noindent
1\\
2\\
3\\
4\\
5\\
6\\
7\\
8\\
9\\
10\\
1\\
2\\
3\\
4\\
5\\
6\\
7\\
8\\
9\\
20\\
1\\
2\\
3\\
4\\
5\\
6\\
7\\
8\\
9\\
30\\
1\\
2\\
3\\
4\\
5\\
6\\
7\\
8\\
9\\
40\\
1\\
2\\
3\\
4\\
5\\
6\\
7\\
8\\
9\\
50\\
1\\
2\\
3\\
4\\
5\\
6\\
7\\
8\\
9\\
60\\
1\\
2\\
3\\
4\\
5\\
6\\
7\\
8\\
9\\
70\\

\end{verbatim}
の要領です。

\printbibliography[title=参考文献]

\end{document}
